\documentclass{article}

\usepackage{tabularx}
\usepackage{booktabs}
\usepackage{soul,color}
\usepackage{xcolor}

\title{SE 3XA3: Problem Statement\\super-refactored-mario-python}

\author{Team 203, Abstract Connoiseurs
		\\ Daniel Noorduyn, noorduyd
		\\ David Jandric, jandricd
		\\ Alexander Samaha, samahaa
}

\date{}

%\input{../Comments}%

\begin{document}

\begin{table}[hp]
\caption{Revision History} \label{TblRevisionHistory}
\begin{tabularx}{\textwidth}{llX}
\toprule
\textbf{Date} & \textbf{Developer(s)} & \textbf{Change}\\
\midrule
2020/01/22 & Entire Team & Initial Document Creation\\
2020/01/22 & Alexander Samaha & Added finishing touches to paragraph\\
\textcolor{red}{2020/04/06} & \textcolor{red}{Daniel Noorduyn} & \textcolor{red}{Rev 1 Touch-ups}\\
\bottomrule
\end{tabularx}
\end{table}

\newpage

\maketitle

The original Super Mario Bros. took the world by storm in the summer of 1983.
Since then, it has been adored by fans around the world. Sadly, the game was
only accessible on the Nintendo Entertainment System, or by using an emulator
for that system. Futhermore, the game itself, was not modifiable without a deep
understand of the gaming system, and low level assembly code. This is assuming
you could even access the original code, considering it was never open-source.\\

As the years went on, people have tried, and failed, to port the game to various
programming languages and platforms. The goal of this project is to refactor and
finish developing the authentic game that the original owner had started and
further add features to improve the gameplay. \textcolor{red}{The final game
will be playable on any personal computer running python3 and the corresponding
libaries pygame and scipy. The supported operating systems will be Linux based
OS such as Ubuntu, MacOS and Windows.} The stakeholders for this project are
those who would enjoy playing 2D platforming games and like to have the
experience of playing Super Mario Bros. on a modern platform. The project is
open source, and inherently affects other developers in this field. \\

This problem is important to our stakeholders due to the cultural significance
of the game, having catapulted the concept into a series that still exists 35
years later. Super Mario Bros. and the franchise as a whole has inspired
generations of 2D platform games and it is important to continue this tradition.
The game is currently inaccessible to a wider audience that is not knowledgeable
about console emulator programs or do not have the original Nintendo
Entertainment Systems. By realizing this project, we are creating an easier
method for people to play the game that so many cherished when it released. In
doing so, a fresh twist of Super Mario Bros. can be enjoyed by anyone, anywhere,
on any modern platform.

%\wss{comment}%

%\ds{comment}%

%\mj{comment}%

%\cm{comment}%

%\mh{comment}%

\end{document}
