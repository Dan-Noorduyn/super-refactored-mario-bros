\documentclass[12pt, titlepage]{article}

\usepackage{booktabs}
\usepackage{tabularx}
\usepackage{hyperref}
\hypersetup{
    colorlinks,
    citecolor=black,
    filecolor=black,
    linkcolor=red,
    urlcolor=blue
}
\usepackage[round]{natbib}

\title{SE 3XA3: Software Requirements Specification\\super-refactored-mario-python}

\author{Team 203, Abstract Connoiseurs
		\\ Alexander Samaha, samahaa
		\\ Daniel Noorduyn, noorduyd
		\\ David Jandric, jandricd
}

\date{\today}

%\input{../Comments}%

\begin{document}

\maketitle

\pagenumbering{roman}
\tableofcontents
\listoftables
\listoffigures

\begin{table}[bp]
\caption{\bf Revision History}
\begin{tabularx}{\textwidth}{p{3cm}p{2cm}X}
\toprule {\bf Date} & {\bf Version} & {\bf Notes}\\
\midrule
Feb 4 2020 & 1.0 & Initial changes to the document \\
Date 2 & 1.1 & Notes\\
\bottomrule
\end{tabularx}
\end{table}

\newpage

\pagenumbering{arabic}

This document describes the requirements for super-refactored-mario-python, which is an implementation of the Super Mario Bros. game in Python. The template for the Software
Requirements Specification (SRS) is a subset of the Volere
template~\citep{RobertsonAndRobertson2012}. This document serves to inform users and stakeholders of the overview of the project including constraints and objectives this project hopes to meet.

\section{Project Drivers}

\subsection{The Purpose of the Project}
    The purpose of super-refactored-mario-python is to re implement and modify the age old Super Mario Bros. game in an open source environment. The team will finish the work started and left uncompleted, then look to add features that diverge from the original game. The end result is a version of Super Mario Bros. that can be enjoyed by anyone with access to a computer with the proper libraries installed.
    
\subsection{The Stakeholders}

\subsubsection{The Client}
    The client for super-refactored-mario-python is the instructor and teaching assistants of the SFWRENG 3XA3 course.
\subsubsection{The Customers}
    The customers of this project are all with an interest in the original Super Mario Bros. game and open-source game development in general. This project will be available for anyone to download if they have access to a computer that satisfies the hardware and software requirements.
\subsubsection{Other Stakeholders}
    The original project owner has a vested interest in the completion of their unfinished project. This may include the integration of the changes and features put forward during the development cycle. Furthermore, the open-source pyGame community will have a vested interest in the propagation of its use to make video games. Finally, the development team also has a vested interest in the success of the project.
\subsection{Mandated Constraints}
    \textbf{Description:} The project shall use the pyGame library for Python game development.\\
    \textbf{Rationale:} The project is originally written with the pyGame library and will make it easier to continue using it.\\
    \textbf{Fit Criterion:} Running the game files with Python includes and runs the pyGame library.\\\\
    \textbf{Description:} The project shall follow the structure of deliverables outlined in the project schedule.\\
    \textbf{Rationale:} The project needs to follow a logical schedule to ensure completion of the product within the allotted time.\\
    \textbf{Fit Criterion:} The project is completed and all deliverables submitted by April 6th 2020.
\subsection{Naming Conventions and Terminology}
    \textbf{SFWRENG 3XA3} - The Software Engineering Practice and Experience: Software Project Management course instructed by Dr. Asghar Bokhari.\\
    \textbf{Super Mario Bros.} - The original arcade game released in 1985 for the Nintendo Entertainment System in North America.\\
    \textbf{pyGame} - Popular library for small open-source games written in Python.\\
\subsection{Relevant Facts and Assumptions}
    It is assumed that the user has access to a modern computer with the necessary Python interpreter installed along with the pyGame library and is able to run the files. Along with this, the user is expected to have working knowledge of running Python code from the command line. The user is assumed to have a keyboard and a mouse to be able to interact with the game. It is assumed that the user has installed all the required artefacts in a compatible operating system environment.
User characteristics should go under assumptions.

\section{Functional Requirements}

\subsection{The Scope of the Work and the Product}

\subsubsection{The Context of the Work}

\subsubsection{Work Partitioning}

\subsubsection{Individual Product Use Cases}

\subsection{Functional Requirements}

\section{Non-functional Requirements}

\subsection{Look and Feel Requirements}

\subsection{Usability and Humanity Requirements}

\subsection{Performance Requirements}

\subsection{Operational and Environmental Requirements}

\subsection{Maintainability and Support Requirements}

\subsection{Security Requirements}

\subsection{Cultural Requirements}

\subsection{Legal Requirements}

\subsection{Health and Safety Requirements}

This section is not in the original Volere template, but health and safety are
issues that should be considered for every engineering project.

\section{Project Issues}

\subsection{Open Issues}

\subsection{Off-the-Shelf Solutions}

\subsection{New Problems}

\subsection{Tasks}

\subsection{Migration to the New Product}

\subsection{Risks}

\subsection{Costs}

\subsection{User Documentation and Training}

\subsection{Waiting Room}

\subsection{Ideas for Solutions}

\bibliographystyle{plainnat}

\bibliography{SRS}

\newpage

\section{Appendix}

This section has been added to the Volere template.  This is where you can place
additional information.

\subsection{Symbolic Parameters}

The definition of the requirements will likely call for SYMBOLIC\_CONSTANTS.
Their values are defined in this section for easy maintenance.


\end{document}