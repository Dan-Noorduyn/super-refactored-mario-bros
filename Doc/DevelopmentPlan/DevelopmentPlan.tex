\documentclass{article}

\usepackage{booktabs}
\usepackage{tabularx}
\usepackage{caption}
\usepackage{hyperref}
\usepackage{xcolor}
\usepackage{ulem}

\title{SE 3XA3: Development Plan\\super-refactored-mario-python}

\author{Team 203, Abstract Connoiseurs\\
Daniel Noorduyn, noorduyd\\
David Jandric, jandricd\\
Alexander Samaha, samahaa\\
}

\date{}

%\input{../Comments}%

\begin{document}

\begin{table}[hp]
\caption{Revision History} \label{TblRevisionHistory}
\begin{tabularx}{\textwidth}{llX}
\toprule
\textbf{Date} & \textbf{Developer(s)} & \textbf{Change}\\
\midrule
Jan 26, 2020 & Alexander Samaha & Set-up document, initial changes\\
Jan 28, 2020 & Daniel Noorduyn & POC and Roles Rough Draft\\
Jan 28, 2020 & Alexander Samaha & Technology and Communication Rough Draft and Final\\
Jan 29, 2020 & Daniel Noorduyn & POC and Roles Touched Up\\
Jan 29, 2020 & David Jandric & TMP, GWP, CS Rough Draft and Final\\
Jan 29, 2020 & Alexander Samaha & Project Schedule completed and linked\\
Jan 31, 2020 & Daniel Noorduyn & Finshing Touches\\
\textcolor{red}{April 6, 2020} & \textcolor{red}{Daniel Noorduyn} & \textcolor{red}{Rev 1 Touch-ups}\\
\bottomrule
\end{tabularx}
\end{table}

\newpage

\maketitle

This document lays out the plan for developing, testing and reviewing our
project.

\section{Team Meeting Plan}

Meetings will take place every Tuesday and Wednesday in \sout{ITB 236 from
2:30PM to 4:30PM} \textcolor{red}{Thode Library from 4:30PM to 5:30 PM}.
Informal meetings will be held on Thursdays at 6:00PM, if required. Alex will
log, Dan will be chair, David will act as technology expert.
Rules for agendas:
\begin{enumerate}
    \item Items must be pertinent to next deliverable.
    \item All team members should have some input into the direction of the meetings.
    \item Items in agenda should be posed as questions.
    \item Give times for each item.
    \item Agenda should be shared with each member a day before the meetings.
\end{enumerate}

\section{Team Communication Plan}

The development group have all exchanged Facebook Messenger credentials, phone
numbers and academic email handles. This allows the team three channels of
communication outside of the predetermined weekly meetings.

The team has set up a Facebook Messenger group chat for discussing work related
activities among all group members. Messenger provides an important avenue to
discuss minor issues related to the project, in contrast, the weekly meetings
will be used to discuss major issues and activities in detail. This could
involve current bugs or issues with the program, blockers where a team member is
unsure how to proceed, as well as current updates on development of features.
Casual work discussion not fitting the previous criteria would also be
appropriate on Facebook Messenger. Update conference calls will also happen on
this platform through the conference call functionality. This should occur
before a development sprint and throughout the development cycle to update team
members on features to be completed and manage a prioritization schedule.

More time sensitive issues involving individual team members will be pursued
through telephone communication. Phone calls or text messages would be
appropriate to communicate issues such as critical blockers that need to be
fixed.

The team members have access to GitLab through our academic emails. This
directly connects the team to the source code-base to communicate merge requests
and issues. Real-time or synchronous messages will be avoided with this channel
since it is not likely to elicit a fast response relative to phone and Facebook
Messenger communication. Furthermore, our academic emails open a direct channel
between our team and supervising TA. This will be used to pass information both
ways regarding different stages of the project and feedback on progress.

\section{Team Member Roles}

A project with this level of coordination required amidst developers and
designers results in the need for a chair (or team leader). The chair's
responsibility is making sure each team meeting runs smoothly within the
allotted time frame and makes sure each team member is communicated with each
other during the designing and building phases. The team has democratically
voted in Daniel Noorduyn to fulfill this position.  During the team meetings, a
scribe will be required to jot minutes down. These minutes will be referred to
by individual team members to recall all team decisions during individual work.
The team has democratically voted in Alexander Samaha. During the design phase,
a technology lead is required to determine if certain project aspects are
feasible. The team has voted that David Jandric fulfill this position.

Each member of the team has unique expertise that will benefit the overall
project. The breakdown is in the following table:

\textcolor{red}{Old table:}

\begin{center}
  \begin{tabular}{ |c|c| }
    \hline
    \sout{Team Member} & \sout{Expertise}\\
    \hline
     \sout{David Jandric} & \sout{Documentation}\\
     \sout{Alexander Samaha} & \sout{Git}\\
     \sout{Daniel Noorduyn} & \sout{LaTeX}\\
     \sout{David Jandric} & \sout{Technology}\\
     \sout{Alexander Samaha} & \sout{Design}\\
    \hline
  \end{tabular}
\end{center}

\textcolor{red}{New table:}

\begin{center}
  \begin{tabular}{ |c|c| }
    \hline
    \textcolor{red}{Team Member} & \textcolor{red}{Expertise}\\
    \hline
    \textcolor{red}{David Jandric} & \textcolor{red}{Documentation}\\
     & \textcolor{red}{Technology}\\
    \hline
    \textcolor{red}{Alexander Samaha} & \textcolor{red}{Git}\\
     & \textcolor{red}{Design}\\
    \hline
    \textcolor{red}{Daniel Noorduyn} & \textcolor{red}{LaTeX}\\
    \hline
  \end{tabular}
  \captionof{table}{Team Member Expertise}
\end{center}

These roles and expertise fields are subject to change as different challenge
arise during the project. Each team member will be filling in multiple roles
from the start, and therefore understand how fluid the assigned roles are.

\section{Git Workflow Plan}

Our repository is centralized around the master branch, and new branches are
created for different features and modules. \textcolor{red}{This type of plan is
called a
\href{https://www.atlassian.com/git/tutorials/comparing-workflows/feature-branch-workflow}{Feature
Branch} workflow plan.} \sout{These branches are} \textcolor{red}{Each branch
is} destroyed once merged with master. Labels will be used in order to track
bugs, stylistic changes, documentation requirements, issues that may require
discussion, and suggestions. We will be using milestones in order to track the
due dates of the deliverables.

\section{Proof of Concept Demonstration Plan}

The ambitious goal of creating a Super Mario Bros game that is accessible on
every platform results in some possible hurdles that will need to be overcome if
the project is seen as feasible. We the team want to create a game that blends
elements from the original Super Mario Bros with features from the newer
versions. However, considering the popularity of the game, both past and
present, users will have strong opinions as to what features the game should
possess and how they should be implemented. Therefore, a minimum viable product
(MVP) will be created that users can interact with and critique. This feedback
will help us continue creation of the game in a way that results in a wanted
final product.

There are also some strictly technological issues that will have to be
addressed. Our game will be created using PyGame, a Python library. This library
is difficult to install and run properly on an Apple Computer. Therefore, even
though Python is compatible with most operating systems, portability with
respect to Apple computers will be an issue due to the incompatible library. The
implementation of our game is quite straightforward and feasible, but
difficulties may arise when many different methods must work together in real
time to create a seamless game-play for the user.

There will be difficulties in testing our product as it relies heavily on human
input. Therefore, multiple individuals are needed to play the game and report
any bugs. It is important to note that not all testing will be done by external
players, as different methods and algorithms can be tested individually. This
logical testing will be performed using PyTest. However, for testing overall
interaction of the program attributes, we will have to look outside ourselves.

To present proof that the previously mentioned risks can be overcome and the game can
be developed, we will have a demonstration. To underline that the library issue
was overcome, this demonstration will be run on a Mac-book. The demonstration
will consist of a 2D plane where in a character can be controlled to move left
and right as well as jump up and fall down. If this can be achieved the overall
project should be able to be completed.

\section{Technology}

\subsection*{Programming Language:}

The programming language to be used for the project is Python. We will be taking
advantage of the PyGame library which is beneficial for rapid prototyping and
smaller scope games, it is also the most widely recognized game development
library through Python. The game we are producing does not require intensive
resources and as such, Python utilizing the PyGame library would be adequate to
satisfy the game’s requirements. The original project is already utilizing this
language and library which allows the team to efficiently continue from what was
left off. This solution allows for an easy set-up for users of the project,
requiring the Python interpreter to be installed along with a version of PyGame
through the pip installer.

\subsection*{IDE:}

There will be no mandatory IDE to develop the project, however, the preferred
development environment will be Microsoft VSCode. Although not a dedicated
Python development platform, VSCode provides many advantages that make
development more efficient and convenient. VSCode houses tools that perform
linter checks and code completion for Python, helping the development team
release higher quality and consistent code. VSCode also has terminal support
which speeds up the process of testing the code since it can be run in-app. The
chosen IDE also supports Git integration, which encourages best practices such
as branching and frequent code committing to the source code repository.

\subsection*{Testing Framework:}

Testing during the course of the project will be achieved using pyTest and
manual testing. pyTest is a popular Python testing framework that allows us to
perform dynamic white box testing and ensuring appropriate code coverage for all
parts of the program. The team already has experience using this framework which
makes it easier to implement for this solution. Manual testing will be done to
ensure that in-game assets and game-play are behaving as expected when run.

\subsection*{Documentation:}

The project will be documented in full using LaTeX and made available for all
users with access to see on GitLab. LaTeX allows the development team to create
efficient and consistent documents that can be easily compiled into PDF format.
All development team members have extensive use of LaTeX in previous projects.
For code documentation, proper development practices related to code commenting
will be followed. In-line comments will be used to describe the function and use
of the sequence of code. Further, Doxygen will be used to encompass the
functionality of the entire source code base. This will allow us to track the
logic of each class, method and field across the whole project and act as a
reference for future development.

\section{Coding Style}

Since our project will be coded in Python, we will be following the
\href{https://www.python.org/dev/peps/pep-0008/}{\textcolor{red}{PEP8}} style standard for
Python, which comes from the creators of Python. A coding style like this can be
helpful by keeping the structure and look of our code consistent, which helps
with maintaining the code base. Using a consistent coding style can also help
other developers read and understand the code.

\section{Project Schedule}
A link to our project schedule can be found in the project repository linked below:

\href{https://gitlab.cas.mcmaster.ca/jandricd/super-refactored-mario-bros/tree/master/ProjectSchedule}{Link to Gantt chart and Resource chart}

{\it This Gantt chart is updated throughout the project}


\section{Project Review}
Not Applicable.

\end{document}
