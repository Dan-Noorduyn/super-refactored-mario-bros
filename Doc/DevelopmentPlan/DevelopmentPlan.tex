\documentclass{article}

\usepackage{booktabs}
\usepackage{tabularx}
\usepackage{caption}

\title{SE 3XA3: Development Plan\\Title of Project}

\author{Team 203, Abstract Connoiseurs\\
Daniel Noorduyn, noorduyd\\
David Jandric, jandricd\\
Alexander Samaha, samahaa\\
}

\date{}

%\input{../Comments}%

\begin{document}

\begin{table}[hp]
\caption{Revision History} \label{TblRevisionHistory}
\begin{tabularx}{\textwidth}{llX}
\toprule
\textbf{Date} & \textbf{Developer(s)} & \textbf{Change}\\
\midrule
Jan 26, 2020 & Alexander Samaha & Set-up document, initial changes\\
Jan 28, 2020 & Daniel Noorduyn & POC and Roles Rough Draft\\
Jan 29, 2020 & Daniel Noorduyn & POC and Roles Touched Up\\
Jan 29, 2020 & David Jandric & TMP, GWP, CS Rough Draft\\
\bottomrule
\end{tabularx}
\end{table}

\newpage

\maketitle

Put your introductory blurb here.

\section{Team Meeting Plan}
% By: David
% Need to include :  When?  Where?  Frequency?  Roles?  Rules for agendas?
% Rubric:  5 marks.

Meetings will take place every Tuesday and Wednesday in ITB 236 from 2:30PM to
4:30PM. Informal meetings will be held on Thursdays at 6:00PM, if required.
Alex will log, Dan will be chair, David will act as technology expert.

Rules for agendas:
\begin{enumerate}
    \item Items must be pertinent to next deliverable.
    \item All team members should have some input into the direction of the meetings.
    \item Items in agenda should be posed as questions.
    \item Give times for each item.
    \item Agenda should be shared with each member a day before the meetings.
\end{enumerate}

\section{Team Communication Plan}
By: Alex

How we communicate as a team.  Git issues (through merge requests and wiki
posts?), email, group chats, phones.  What contact information is exchanged and
what is used for what information (issues through git,  updates on groupchats,
phone for meetings?)

Rubric:  3 marks.

\section{Team Member Roles}
% By: Dan
% Name  a  team  leader,  scribe.   Then  we  identify  experts  based  on  git,
% LATEX,
% technology,  perhaps  domain.   Roles  are  fluid  and  you  should  know
% which  hat you wear.
%
% Rubric:  2 marks


A project with this level of coordination required amidst developers and
designers results in the need for a chair (or team leader). The chair's
responsibility is making sure each team meeting runs smoothly within the
allotted time frame and makes sure each team member is communicated with each
other during the designing and building phases. The team has democratically
voted in Daniel Noorduyn to fulfill this position.  During the team meetings, a
scribe will be required to jot minutes down. These minutes will be referred to
by individual team members to recall all team decisions during individual work.
The team has democratically voted in Alexander Samaha. During the design phase,
a technology lead is required to determine if certain project aspects are
feasible. The team has voted that David Jandric fulfill this position.

Each member of the team has unique expertise that will benefit the overall
project. The breakdown is in the following table:

\begin{center}
  \begin{tabular}{ |c|c| }
    \hline
    Team Member & Expertise\\
    \hline
     David Jandric & Documentation\\
     Alexander Samaha & Git\\
     Daniel Noorduyn & LaTeX\\
     David Jandric & Technology\\
     Alexander Samaha & Design\\
    \hline
  \end{tabular}
  \captionof{table}{Team Member Expertise}
\end{center}

These roles and expertise fields are subject to change as different challenge
arise during the project. Each team member will be filling in multiple roles
from the start, and therefore understand how fluid the assigned roles are.

\section{Git Workflow Plan}
% By: David 
% Pretty  simple,  how  are  our  branches  organized,  is  our  repository  
% centralized?
% How do we label everything, how will milestones be used?
% Rubric:  3 marks

Our repository is centralized around the master branch, and new branches are 
created for different features and modules. These branches are destroyed once merged with master. 

\section{Proof of Concept Demonstration Plan}
% By: Dan
% Validate technical feasibility.
% https://sensinum.com/proof-of-concept-in-software-development/
%
% Identify risks, is testing going to be difficult?  what are difficult
% implementa-
% tion aspects.  Are there required libraries that have some concerns (pygame and
% MacOS for example).  Will portability be a concern (yes for new Macs I guess,
% out  of our control but lets talk about it).
%
% What will we demonstrate to overcome the risks.  (talk about what we might
% show in terms of testing, what our product will be?  demos of levels?).
%
% Rubric:  5  marks.   Gonna  be  a  biggie.   Maybe  we  have  a  table  here?
% some figures.


The ambitious goal of creating a Super Mario Bros game that is accessible on
every platform results in some possible hurdles that will need to be overcome if
the project is seen as feasible. We the team want to create a game that blends
elements from the original Super Mario Bros with features from the newer
versions. However, considering the popularity of the game, both past and
present, users will have strong opinions as to what features the game should
possess and how they should be implemented. Therefore, a minimum viable product
(MVP) will be created that users can interact with and critique. This feedback
will help us continue creation of the game in a way that results in a wanted
final product.

There are also some strictly technological issues that will have to be
addressed. Our game will be created using pygame, a Python library. This library
is difficult to install and run properly on an Apple Computer. Therefore, even
though Python is compatible with most operating systems, portability with
respect to Apple computers will be an issue due to the incompatible library. A
possible solution is creating an exe file for the users to run, which bypasses
the need for users to download libraries. The implementation of our game is
quite straightforward and feasible, but difficulties may arise when many
different methods must work together in real time to create a seemless gameplay
for the user.

There will be difficulties in testing our product as it relies heavily on human
input. Therefore, multiple individuals are needed to play the game and report
any bugs. It is important to note that not all testing will be done by external
players, as different methods and algorithms can be tested individually. This
logical testing will be performed using PyTest. However, for testing overall
interaction of the program attributes, we will have to look outside ourselves.

To present proof that the afore-mentioned risks can be overcome and the game can
be developed, we will have a demonstration. To underline that the libary issue
was overcome, this demonstration will be run on a MacBook. The demonstration
will consist of a 2D plane where in a character can be controlled to move left
and right as well as jump up and fall down. If this can be achieved the overall project should be able to be completed.

\section{Technology}
By: Alex
Talk  about  the  Programming  languages,  are  we  using  an  IDE?  Testing
frame-
work,  how  are  we  testing  (maybe  something  to  do  with  pytest).
Document
generation, how are we making documentation in Python?

Rubric:  4 marks, we need to write a bit to really bring this one out.

\section{Coding Style}
By: David
% We  should  follow  a  coding  style,  this  can  involve  but  it’s  important 
% that  the code is consistent.  Someone needs to research some python coding 
% styles.  Then we talk about it and how it can be helpful!
% Rubric:  2  marks.   
% Should  be  enough  to  just  introduce  the  style  and what
% advantages it brings.

Since we our project will be coded in Python, we will be following the PEP8 
standard for Python, which is a style standard from the creators of Python.
A coding style like this can be helpful by keeping the structure and look of 
our code consistent, which helps with maintaining the code base.



\section{Project Schedule}
By: All
Provide a pointer to your Gantt Chart.

Rubric:  5 marks.  Needs to be detailed and accurate (realistic).

\section{Project Review}
By: All
Looks like this is necessary for our revisions.

Rubric:  1 mark

\end{document}
